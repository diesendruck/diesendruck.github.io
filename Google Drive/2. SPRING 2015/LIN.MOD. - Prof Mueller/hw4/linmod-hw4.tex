\documentclass[11pt,letterpaper]{article}
\usepackage[utf8]{inputenc}
\usepackage[margin=.5in]{geometry}
\usepackage{graphicx}
\usepackage{titling}
\usepackage{amsmath}
\usepackage{amsfonts}
\usepackage{amssymb}
\setlength{\droptitle}{-6em}
\author{Maurice Diesendruck, Sukyung Park, Bowei Yan, Michael Zhang}
\title{SDS 387 HW 4}
\usepackage{Sweave}
\begin{document}
\Sconcordance{concordance:linmod-hw4.tex:linmod-hw4.Rnw:%
1 11 1 1 0 126 1 4 0 2 3 2 1}

\maketitle
\setcounter{section}{2}

\section{Comparison}
This is an analysis of the LASSO, BLASSO, and Horseshoe priors on diabetes data.
  \begin{enumerate}    
      \item LASSO \\
      The LASSO is an L1-penalized least squares estimate of $\beta$, with $\tilde{y}$
      as the mean-centered outcome variable, where the following is optimized:\\
      $$\min_{\beta} (\tilde{y} - XB)'(\tilde{y} - XB) + 
        \lambda\sum\limits_{j=1}^p |\beta_{j}|$$
      \begin{enumerate}
        \item Strengths
              \begin{enumerate}
                \item Easy to implement.
              \end{enumerate}
        \item Weaknesses
              \begin{enumerate}
                \item Proposed standard error estimators are not considered satisfactory.
              \end{enumerate}
      \end{enumerate}
      
      \item BLASSO \\
      The BLASSO interprets the L1-penalty as a prior on $\beta$ and $\sigma^2$. In this 
      case, $y$ is normal, the prior $\pi(\beta|\sigma^2)$ is Laplace (i.e. double
      exponential), and the prior $\pi(\sigma^2)$ is non-informative and defined as such to 
      ensure a 
      unimodal posterior:
      \begin{align*}
      y|\mu,X,\beta,\sigma^2 &\sim\ N(\mu1_n+X\beta,\sigma^2I_n) \\
      \pi(\beta|\sigma^2)&=\prod\limits_{j=1}^p \frac{\lambda}{2\sqrt{\sigma^2}}e^{-\lambda
        |\beta_j|/\sqrt{\sigma^2}} \hspace{5mm}\text{(Laplace distribution)}\\
      \pi(\sigma^2)&=\frac{1}{\sigma^2}
      \end{align*}
      \begin{enumerate}
        \item Strengths
              \begin{enumerate}
                \item Easy to implement
                \item Automatically provides interval estimates for all parameters, 
                  including the error variance.
              \end{enumerate}
        \item Weaknesses
              \begin{enumerate}
                \item More computationally intensive than LASSO.
              \end{enumerate}
      \end{enumerate}
      
      \item Horseshoe \\
      The Horseshoe (HS) prior assumes $y|\theta \sim\ N(\theta,\sigma^2 I)$, and aims to
      (1) estimate $\theta$, and (2) predict future realizations of $y$. It is especially
      useful in cases where most covariates are nearly zero (i.e. sparse $\theta$). The
      model is as follows:
      \begin{align*}
      \theta_i|\lambda_i &\sim\ N(0,\lambda_i^2)\\
      \lambda_i|\tau &\sim\ C^+(0,\tau)\\
      \tau &\sim\ C^+(0,\sigma)\\
      E(\theta_i|y) &= \int_{0}^{1} (1-\kappa_i)y_ip(\kappa_i|y)\text{d}\kappa_i = (1-E
      (\kappa_i|y)) y_i
      \end{align*}
      Where $\kappa_i = 1/(1+\lambda_i^2)$, assuming fixed values $\sigma^2=\tau^2=1$. Given
      $\lambda_i \geq 0$, this implies that $\lambda_i$'s are indirectly related to amount
      of shrinkage (e.g. high $\lambda_i$ means low $\kappa_i$, and less shrinkage).
      \begin{enumerate}
        \item Strengths
              \begin{enumerate}
                \item Easy to implement.
                \item Has fewer hyperparameters.
                \item Robust and flexible to high or low sparsity situations.
                \item Converges efficiently.
              \end{enumerate}
        \item Weaknesses
              \begin{enumerate}
                \item 
                \item 
              \end{enumerate}
      \end{enumerate}
  \end{enumerate}

\section{Perform Gibbs Sampling for Bayesian Lasso}
  Reference: Park and Casella (2008) (Section 2).
  \begin{enumerate}
      \item Model and Prior for BLASSO\\
      See 3.2, above.
      \item Complete Conditional for each (set of) Parameters\\
      To get complete conditionals, the prior $\pi(\beta|\sigma^2)$ is represented
      differently, as a scaled mixture of normals (with exponential mixing density).
      Again, the Laplace can be re-written as the integrated product of a Normal and
      Exponential distribution:\\
          \begin{align*}
          \pi(\beta|\sigma^2)&=\prod\limits_{j=1}^p \frac{\lambda}{2\sqrt{\sigma^2}}e^{
          -\lambda|\beta_j|/\sqrt{\sigma^2}} \hspace{33mm}\text{(Laplace distribution)}\\
          &=\prod\limits_{j=1}^p \int_{0}^{\infty} \text{ Normal * Exponential}
          \end{align*}
      This way, full conditionals become:\\
          \begin{align*}
          y|\mu,X,\beta,\sigma^2 &\sim\ N(\mu1_n+X\beta,\sigma^2I_n),\\
          \beta|\sigma^2,\tau_1^2,\ldots,\tau_p^2 &\sim\ N_p(0_p,\sigma^2D_\tau),
            &&\hspace{5mm}\text{(Normal part of Laplace)}\\
          \text{ where }D_\tau &= diag(\tau_1^2,\ldots,\tau_p^2),\\
          \sigma^2,\tau_1^2,\ldots,\tau_p^2 &\sim\ \pi(\sigma^2)d\sigma^2 \prod_{j=1}^p
            \frac{\lambda^2}{2}e^{\lambda^2\tau_j^2/2}d\tau_j^2,
            &&\hspace{5mm}\text{(Exponential part of Laplace)}\\
          \sigma^2,\tau_1^2,\ldots,\tau_p^2 &> 0.
          \end{align*}
      
  \end{enumerate}












\newpage



\subsection*{Results and Discussion}
Our model is run with the following command:
\begin{Schunk}
\begin{Sinput}
> # R code here!
\end{Sinput}
\end{Schunk}



\end{document}
